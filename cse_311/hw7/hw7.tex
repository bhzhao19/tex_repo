%%%%%%%%%%%%%%%%%%%%% PACKAGE IMPORTS %%%%%%%%%%%%%%%%%%%%%
\documentclass[11pt]{article}
\usepackage{amsmath, amsfonts, amsthm, amssymb}
\usepackage{lmodern}
\usepackage{microtype}
\usepackage{fullpage}
\usepackage{algorithm}
\usepackage[noend]{algpseudocode}
\algrenewcommand\textproc{} % \textsc is an alternative

\usepackage[x11names, rgb]{xcolor}
\usepackage{graphicx}
% \usepackage{graphicx, wrapfig,fullpage}
\usepackage{circuitikz} % tikz
\usetikzlibrary{decorations,arrows,shapes}
% Converts PDF viewer to dark mode (comment the below two lines to revert back before submitting HW)
% \usepackage{xcolor} \pagecolor[rgb]{0,0,0} \color[rgb]{1,1,1} % Normal dark mode

\usepackage{etoolbox}
\usepackage{enumerate}
\usepackage{enumitem}
\usepackage{listings}
\usepackage{array}
\usepackage[most]{tcolorbox}
\usepackage{hyperref}

% Boxes text in a nice way
\tcbset{
    mybox/.style={
        enhanced, % Enable advanced features
        colframe=blue!10, % Border color
        boxrule=0.5pt, % Border width
        leftrule=0.5pt, % Left border width
        rightrule=0.5pt, % Right border width
        toprule=0.5pt, % Top border width
        bottomrule=0.5pt, % Bottom border width
        width=\linewidth,
        enlarge left by=0mm,
        boxsep=5pt,
        arc=0pt,outer arc=0pt,
        breakable,
        sharp corners,
    }
}

% Alternate: Boxes text with a blue border 
% \tcbset{
%     frame code={},
%     colback=blue!10,
%     leftrule=0.5pt,
%     rightrule=0.5pt,
%     toprule=0.5pt,
%     bottomrule=0.5pt,
%     width=\linewidth,
%     enlarge left by=0mm,
%     boxsep=5pt,
%     arc=0pt,outer arc=0pt,
%     breakable,
%     colframe=white,
%     sharp corners
% }

\hbadness=10000 % Underfull upperbound
\hfuzz=95pt % Overfull upperbound

%%%%%%%%%%%%%%%%%%%%%%%% QUESTION # %%%%%%%%%%%%%%%%%%%%%%%%
%% This part helps number your questions and creates a    %%
%% new page with each question for the aesthetics.        %
%% It also creates parts that helps you answer questions  %%
%% To use do: \begin{question} ... \end{question}         %%
%% and: \begin{part} ... \end{part}                       %%
%%%%%%%%%%%%%%%%%%%%%%%%%%%%%%%%%%%%%%%%%%%%%%%%%%%%%%%%%%%%
\setlength{\parindent}{0pt}
\setlength{\parskip}{5pt plus 1pt}

\providetoggle{questionnumbers}
\settoggle{questionnumbers}{true}
\newcommand{\noquestionnumbers}{
    \settoggle{questionnumbers}{false}
}

\newcounter{questionCounter}
\newenvironment{question}[2][\arabic{questionCounter}]{%
    \addtocounter{questionCounter}{1}%
    \ifnum\value{questionCounter}=1 {} \else {\newpage}\fi
    \setcounter{partCounter}{0}%
    \vspace{.25in} \hrule \vspace{0.5em}%
        \noindent{\bf \iftoggle{questionnumbers}{Question #1: }{}#2}%
    \vspace{0.8em} \hrule \vspace{.10in}%
}

\newcounter{partCounter}[questionCounter]
\renewenvironment{part}[1][\alph{partCounter}]{%
    \addtocounter{partCounter}{1}%
    \vspace{.10in}%
    \begin{indented}%
       {\bf (#1)} %
}{\end{indented}}

\def\indented#1{\list{}{}\item[]}
\let\indented=\endlist
\def\show#1{\ifdefempty{#1}{}{#1\\}}

%%%%%%%%%%%%%%%%%%%%%%%% SHORT CUTS %%%%%%%%%%%%%%%%%%%%%%%%
%% This is just to improve your quality of life. Instead  %%
%% of having to type long things, you can type short      %%
%% things. Ex: \IMP instead of \rightarrow to get ->      %%
%%%%%%%%%%%%%%%%%%%%%%%%%%%%%%%%%%%%%%%%%%%%%%%%%%%%%%%%%%%%
\def\IMP{\rightarrow}
\def\AND{\wedge}
\def\OR{\vee}
\def\BI{\leftrightarrow}
\def\DIFF{\setminus}
\def\SUB{\subseteq}
\def\code#1{{\texttt{#1}}}
\def\pred{\code}
\newcommand{\trees}{\textbf{Trees}}
\newcommand{\tree}{\textbf{Tree}}
\newcommand{\height}{\operatorname{height}}
\newcommand{\leaves}{\operatorname{leaves}}
\newcommand{\Mod}[1]{\ (\mathrm{mod}\ #1)}
\newcommand{\ds}{\displaystyle}

\newcolumntype{C}{>{\centering\arraybackslash}m{1.5cm}}
\renewcommand\qedsymbol{$\blacksquare$}

\title{\vspace{-2cm}Homework 7: Structural Induction, Regexes, CFGs \vspace{-0.3cm}}
\author{Collaborators: List collaborators here\vspace{-0.5cm}}
% Don't put your name above so we don't have implicit bias when grading
\date{\vspace{-0.3cm}\today\vspace{-1cm}}

%%%%%%%%%%%%%%%%%%%%%%%% WORK BELOW %%%%%%%%%%%%%%%%%%%%%%%%
\begin{document}
\maketitle

\begin{question}{Walk the walk, talk the talk [20 points]}
    Let $S$ be a subset of $\mathbb{Z}\times \mathbb{Z}$ defined recursively as:

    \textbf{Basis Step:} $(4,3) \in S$\\
    \textbf{Recursive Step:} if $(a,b) \in S$ then $(a + 5, b + 7) \in S$ and $(a + 11, b + 8)\in S$

    Prove that $\forall (a,b) \in S, a \leq 2b$.

    \textbf{Hint}: Remember that with structural induction you must show $P(s)$ for every element $s$ that is added by the recursive rule -- you will need to show $P()$ holds for two different elements in your inductive step.

    \textbf{Hint}: The inequality you're showing in your inductive step isn't always tight (that is it could be that $<$ is true, not just $\leq$), remember to keep your target in mind!
    \begin{tcolorbox}
        Proof by structural induction: \\
        The proposition we want to proof, $P(s)$, is that: If $s = (x,y) \in S$, then $x \leq 2y$. \\

        \textbf{Base Case:} There is only one base case, which is the basis step starting for the set $S$, where $s_0 = (4,3)$. \\
        Under such case, $P(s_0)$ will be: $4\leq 2\cdot 3 = 6$. Thus we know that $P(s_0)$ is true.\\ 

        Let $y$ be an arbitrary element of $S$ not covered by base case. By the exclusion rule, for all $x = (a,b)\in S$, $y = (a + 5, b + 7)$ or $y = (a + 11, b + 8)$. \\ \\


        \textbf{Induction Hypothesis:} Suppose that $P(x)$ is true, which means that $a \leq 2b$ holds.
        \textbf{Inductive step:} Then there will be two cases for $y\in S$: $y = (a + 5, b + 7)$ or $y = (a + 11, b + 8)$.
        Case 1: if $y = (a + 5, b + 7)$. From the \textbf{IH}, we know that: $a - 2b \leq 0$.
        \begin{align*}
            (a+5) - 2(b+7) &= a-2b + 5 - 14 && \text{Algebra} \\
            &= (a - 2b) - 9 && \text{Algebra} \\
            &\leq 0 - 9 && \text{Replacement by the inequality in IH} \\
            &= -9 && \text{Algebra} \\
            &\leq 0 && \text{Algebra, inequality} 
        \end{align*}
        From the steps above, we can see that $(a+5) - 2(b+7) \leq 0$, which means that $(a+5) \leq 2(b+7)$. Thus, $P(y)$ holds for case 1. \\ \\
        
        Case 2: if $y = (a + 11, b + 8)$. From the \textbf{IH}, we know that: $a - 2b \leq 0$.
        \begin{align*}
            (a+11) - 2(b+8) &= a-2b + 11 - 16 && \text{Algebra} \\
            &= (a - 2b) - 5 && \text{Algebra} \\
            &\leq 0 - 5 && \text{Replacement by the inequality in IH} \\
            &= -5 && \text{Algebra} \\
            &\leq 0 && \text{Algebra, inequality} 
        \end{align*}
        From the steps above, we can see that $(a+11) - 2(b+8) \leq 0$, which means that $(a+11) \leq 2(b+8)$. Thus, $P(y)$ holds for case 2.\\

        Since in both case 1 and case 2, $P(y)$ is true if $P(x)$ holds. Thus, by structural induction, $P(s)$ holds for all $s\in S$.
    \end{tcolorbox}
\end{question}

\begin{question}{Plants [23 points]}
    You are playing an online plant-growing game with a friend. Your plant's health is represented by an ordered pair of two integer values $(w,n)$ where the first number represents how much additional water the plant requires, and the second represents how many additional nutrients it needs. The ideal state for a plant is at $(0,0)$. When it reaches this point, the plant bears fruit.\\
    The plant starts off at some state $(k, k)$ and you and your friend take turns playing the game. In their turn, each player can either water the plant (decreasing the amount of water it needs) or apply manure (decreasing the amount of nutrients it needs), but a player can only do one of those two options (not both). The player gets to pick how much they affect the plant (but its state must change by at least $1$)

    More formally, assuming you are at state $(a,b)$ you can move to any of the states: $(a, b-i)$ where $1 \leq i \leq b$ or $(a-j, b)$ where $1 \leq j \leq a$.

    A player wins the game by being the one who causes the plant to hit $(0,0)$, as that player gets to collect all the fruit! You, generously, allow your friend to go first.

    \begin{enumerate}[label=(\alph*)]
        \item Using induction, prove that in any plant growing game where the plant starts in state $(a,a)$ you (the player that goes second) can always win the game. \\
              Be sure to explicitly and clearly define a predicate $P()$! We \textbf{strongly} recommend your predicate includes the phrase ``It is not my turn'' or ``the second player can'' or something similar.
                  [20 points]
        \begin{tcolorbox}
            Proof by Strong induction: \\
            The predicate we want to proof, $P(n)$, is that: for all positive integer $n$, if the plant is at the state $(n,n)$, the second player can always find a strategy to reach the state $(0,0)$ prior than the first player. \\ \\
            \textbf{Base Case:} The base is at $n = 1$. And $P(1)$ will be: if the plant is at the state $(n,n)$, the second player can always find a strategy to reach the state $(0,0)$ prior than the first player.\\
                In the base case, at the state $(1,1)$, by the game requirement, there are only two possible states that the first player can reach: he will either water the plant, which will 
                reach $(1-j, 1)$ where $1 \leq j \leq 1$, where here $j$ can only be 1, which means that he can only reach $(0,1)$; or, he can choose to apply manure, which will reach 
                $(1, 1-i)$ where $1 \leq i \leq 1$, which means that he can only reach $(1,0)$. \\
                \textbf{Case 1:} If the first user choose to water the plant and reach $(0,1)$. By the requirement of the game, the second user can only apply manure to the plant, for there couldn't exist an 
                integer $i$ such that $1\leq i \leq 0$. Since he can only choose to apply manure, and only manure once, he will finally reach the state $(0,0)$, prior to the first player. Thus, $P(1)$
                is true in case 1.\\
                \textbf{Case 2:} If the first user choose to apply manure to the plant and reach $(1,0)$. By the requirement of the game, the second user can only water the plant, for there couldn't exist an 
                integer $j$ such that $1\leq j \leq 0$. Since he can only choose to water, and only water once, he will finally reach the state $(0,0)$, prior to the first player. Thus, $P(1)$
                is true in case 2.\\
                Since in both cases $P(1)$ is true, $P(1)$ holds. \\ \\
            \textbf{Induction Hypothesis:} Suppose that $k$ is an integer and $k\geq 1$. $P(1)\land ...\land P(k)$ holds. \\
            \textbf{Inductive step:} We want to proof that $P(k+1)$ is true. \\
                In $P(k+1)$, the tree will be at the state of $(k+1,k+1)$. Here the first player has two strategy: he will either water/apply manure with the length of $k+1$, or water/apply manure
                 with length $r\in \mathbb{N}$, where $r$ is in the range of $1\leq r\leq k$. \\
                \textbf{Case 1:} If the first player choose to \textbf{water/apply manure} with the length of $k+1$, the tree will then be at the state of $(0,k+1)$ or $(k+1,0)$. Supposing that the second player is
                really greedy on winning the game, and for either case above, the second player can \textbf{apply manure/water} with the length of $k+1$ corresponding to the another choice that player 1 doesn't choose. And for both 
                cases, the second player can successfully reach the state $(0,0)$ prior then player one, thus, under such case, $P(k+1)$ is true. \\
                \textbf{Case 2:} If the first player choose to \textbf{water/apply manure} with length $r\in \mathbb{N}$, where $r$ is in the range of $1\leq r\leq k$, the second player can also apply the strategy
                of \textbf{apply manure/water} with length $r$ corresponding to the another choice that player 1 doesn't choose. Under such case, the second player can successfully change the state of the plant
                back to $(k+1-r, k+1-r)$. Since $1\leq r\leq k$, $-1\geq -r\geq -k$, and $k+1-1\geq k+1-r\geq k+1-k$, which is the range $1\leq k+1-r\leq k$. And in the \textbf{IH}, we assume that $P(1)\land ...\land P(k)$ holds.
                And the plant at state $(k+1-r, k+1-r)$ will be one of the induction hypothesis. 
        \end{tcolorbox}
        \begin{tcolorbox}
            Since $P(k+1-r)$ holds in our hypothesis, $P(k+1)$ will be ture due to the reason that second player can find a way 
                to reach $(0,0)$ state prior to first player. \\ \\
            Thus, by strong induction we proof that for all positive integer $n$, if the plant is at the state $(n,n)$, the second player can always find a strategy to reach the state $(0,0)$ prior than the first player. \\
            By using the theorm proved above, for any tree state $(a,a)$, the player that goes second (which is me!) will successfully win since I can always reach state $(0,0)$ prior to the first player.
        \end{tcolorbox}

        \item Describe your winning strategy (i.e. describe how you should play the game in order to win, assuming that you go first). A strategy would be something like ``If my friend friend waters the plant for $i$ units then I will...'' [3 points]
        \begin{tcolorbox}
            The strategy is really simple to describe: \\
            If my friend choose to water $n$ steps, then I will choose to apply manure $n$ steps; And if my friend choose to apply manure $n$ steps, then 
            I will choose to water $n$ steps. ($n$ is an arbitrary integer number in the range of $\left[1 , a \right]$)
        \end{tcolorbox}
    \end{enumerate}

\end{question}

\begin{question}{What doesn't kill you makes you stronger [24 points]}
    In this problem, we'll use a technique called ``strengthened induction.'' The idea behind the technique is that sometimes the predicate, $P()$, that you want to prove isn't well-suited to induction. $P()$ might not have enough information so that when you use your inductive hypothesis you don't know enough to show $P(k+1)$. In cases like these, we will sometimes strengthen the predicate.

    Consider the following recursively defined functions:

    \[f(n) = \begin{cases}
            3       & \text{if $n = 1$}                   \\
            3f(n-1) & \text{if $n \in \mathbb{N}, n > 1$} \\
        \end{cases}\]

    \[g(n) = \begin{cases}
            2           & \text{if $n = 1$}                   \\
            3g(n-1) + 2 & \text{if $n \in \mathbb{N}, n > 1$}
        \end{cases} \]

    Let $P(n)$ be ``$g(n) < f(n)$''

    \begin{enumerate}[label=(\alph*)]
        \item Imagine you wanted to prove $P(n)$ for all $n\in \mathbb{N}, n \geq 1$ by induction. Try for about 5-10 minutes to write an inductive proof and see where you get stuck. For this part, just write ``I spent about $X$ minutes trying'' (as long as you report an $X$ at least 5, you will get this point).  [1 point]
            \begin{tcolorbox}
                I spent about 8 minutes trying.
            \end{tcolorbox}
        \item Give an example of numbers $a,b$ such that $a < b$ but $3a+2 \not< 3b$. [1 point]\\
            \textbf{Hint:} You might want to choose non-integer numbers.
                
            \begin{tcolorbox}
                Answer here: \\
                Suppose $a = \frac{1}{3}$ and $b = \frac{1}{2}$. Then, $3a+2 = 1+2 = 3\not< 3\cdot \frac{1}{2} = \frac{3}{2}$.
            \end{tcolorbox}

            Now imagine that for some number $m$, $f(m) = a$ and $g(m) = b$. If there were such a number $m$, it would be very hard to prove the inductive step of this proof! It would be impossible -- from your calculation in part b, $P(m+1)$ would be false then! This observation is an indication that our $P()$ isn't specific enough (or ``strong'' enough, as we'd usually say). Our $P()$ \textbf{is} true for the particular $f,g$ in this problem! But unless our $P()$ ``rules out'' the existence of that value $m$ we won't be able to write the proof.  Take a moment to ponder this observation -- it's the main point of this problem, but also somewhat mind-bending. (Don't write anything for this paragraph.)

        \item Let's try to really prove the claim now. Define the predicate $Q(n)$ to be ``$g(n) \leq f(n) - 1$''. Explain why showing $Q(n)$ for all $n\geq 1$ will also show $P(n)$ for all $n\geq 1$ (1 sentence) [2 points]
            \begin{tcolorbox}
                If we proof $Q(n)$, then we know that for all $n\in \mathbb{N}, n \geq 1$, $g(n) \leq f(n) - 1$. And by doing algebra, we can get that $g(n) + 1 \leq f(n)$. Since we know that $g(n) < g(n) + 1$, 
                we can get that $g(n)<g(n)+1\leq f(n)$, which is the predicate $P(n)$. Thus if shown $Q(n)$ for all $n\geq 1$ then $P(n)$ for all $n\geq 1$ is shown.
            \end{tcolorbox}
        \item Prove $Q(n)$ holds for all integers $n$, with $n\geq 1$ by induction. [20 points]
            \begin{tcolorbox}
            Proof by induction: \\
            The predicate we want to proof, $Q(n)$, is that: ``for all $n \in \mathbb{N}, n\geq 1$, $g(n) \leq f(n) - 1$''\\ \\
            \textbf{Base Case:} When $n=1$, $Q(1)$ will be: \\
                $g(1) = 2$, $f(1) = 3$, $g(1) = 2 \leq 3 - 1 = f(n) - 1$. Thus $Q(1)$ is true. \\ \\
            \textbf{Induction Hypothesis:} Suppose that $k$ is an integer and $k\geq 1$. $Q(k)$ holds. Which means that $g(k) \leq f(k)-1$. We can also rewrite it in the form of $g(k)-f(k)+1 \leq 0$ \\ \\
            \textbf{Inductive step:} We want to proof that $Q(k+1)$ is true, which is $g(k+1) \leq f(k+1)-1$. By applying algebra, the target can be changed to $g(k+1)-f(k+1)+1 \leq 0$.
            \begin{align*}
                g(k+1)-f(k+1)+1 &= (3g(k)+2) - 3f(k)+1 && \text{By the defn of function f and g} \\
                &= 3g(k)-3f(k)+3 && \text{Algebra} \\
                &= 3(g(k)-f(k)+1)&& \text{Algebra, extracting 3 out} \\
                &\leq 3\cdot 0 && \text{Replacing by the inequality in IH} \\
                &= 0 && \text{Algebra} 
            \end{align*}
            Since in the steps above we proved that $g(k+1)-f(k+1)+1 \leq 0$, which means that $g(k+1) \leq f(k+1)-1$. Thus, if $Q(k)$ is true, then we can proof that $Q(k+1)$ is true. \\ \\
            Thus, by mathematical induction, we proved that for all $n \in \mathbb{N}, n\geq 1$, $g(n) \leq f(n) - 1$ is true.
            \end{tcolorbox}
        \item This proof technique is called ``strengthened induction'' (not to be confused with strong vs. weak induction).
              We wanted to show $P()$, but $P()$ wasn't suitable for induction, the I.H. was not enough information to easily prove our target in the I.S.. We defined a strengthened claim, $Q()$. (we call $Q()$ ``stronger'' because once you know $Q()$ is true you also know $P()$ must be true).

              Choosing what to add is a tradeoff -- the more information in $Q()$ that you add, the more information you have in your inductive hypothesis to assume. But also the more you have to show in your inductive step!
              You do not have to write anything for this part. [0 point]
    \end{enumerate}
\end{question}

\begin{question}{Relations [12 points]}
    \begin{enumerate}[label=(\alph*)]
        \item Define the relation $\prec$ on $\mathbb{Z}^+$ as follows: we will say $a \prec b$ if and only if $2a | b$.
              Prove that the relation $\prec$ is transitive  [8 points]\\
              If you're using LaTeX, the symbol in this problem is \texttt{$\backslash$prec}.\\
              \textbf{Hint:} You know how to do this kind of proof! Look at the definition of transitive---it's a for-all statement with an implication inside. You've done tons of proofs like that!
        \begin{tcolorbox}
            Suppose that $a,b,c$ are arbitrary numbers in the set $\mathbb{Z}^+$, and $a\prec b$ and $b\prec c$. By the definition of relation
            $\prec$, we can know that $2a|b$ and $2b|c$. Then, by the definition of divisibility, there exists integers $k_1,k_2$ such that
            $b = 2a k_1$, and $c = 2bk_2$. By substituting b, we can see that $c = 2(2ak_1)bk_2 = (2bk_1k_2) 2a$. Since $b,k_1,k_2$ are integers, 
            by the closure of integer we know that $2bk_1k_2$ is an integer. \\ \\
            Since there exists an integer  $ 2bk_1k_2 $ such that $c = (2bk_1k_2) 2a$, by the integer divisibility, we can conclude that $2a|c$. And then
            by the definition of $\prec$, we know that $a\prec c$. \\ \\
            From the steps above, we proof that for all $a,b,c\in \mathbb{Z}^+$, if $a\prec b$ amd $b\prec c$, then $a \prec c$. Thus, the relation
            $\prec$ on $\mathbb{Z}^+$ satisfied the transitivity.
        \end{tcolorbox}
        \item Disprove the following statement: for all relations $R$, $S$: if $R$ and $S$ are antisymmetric relations on the same set $A$, then the relation $R \cup S$ is antisymmetric on the set $A$ as well. [4 Points]\\
              \textbf{Hint:} Remember that relations are sets of ordered pairs! Since they're sets, we can do set operations (like union) on them. The combined relation has all pairs that are in $R$ or $S$ (or both)\\
              \textbf{Hint:} You know how to do this kind of proof too! You're disproving a for-all. Remember to make your counter-example as specific as possible.
        \begin{tcolorbox}
            Proof by counter example: \\
            Suppoes that set $A = \{1,2\}$, set of relation $R = \{(1,2)\}$, and $S = \{(2,1)\}$. By the definition of antisymmetric, we can tell that
            both the two relation $R$ and $S$ are antisymmetric, for if $1R2$, then $2R1$ must not hold. Similarly, if $2S1$, then $1S2$ must not hold.
            However, the union set $R\cup S = \{(1,2),(2,1)\}$. And we can see that the union set $R\cup S$ is not an antisymmetric relationship on $A$,
            since $1(R\cup S)2$ and $2(R\cup S)1$ holds simutanenously.
        \end{tcolorbox}
    \end{enumerate}
\end{question}

\end{document}
